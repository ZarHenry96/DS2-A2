\documentclass[11pt,twocolumn,letterpaper]{article}

%\usepackage{cvpr}
\usepackage{subfig}
\usepackage{times}
\usepackage{epsfig}
\usepackage{graphicx}
\usepackage{amsmath}
\usepackage{lipsum}
\usepackage{amssymb}
\usepackage[british]{babel}
\usepackage[backend=biber, style=ieee]{biblatex}
\usepackage{url}
\usepackage[bookmarks=true]{hyperref}
\usepackage[noabbrev,capitalise]{cleveref}
\usepackage{appendix}
\usepackage{booktabs}
\bibliography{biblio}

\begin{document}
	
	%%%%%%%%% TITLE
	\title{Chord: A Scalable Peer-to-peer Lookup Protocol for Internet Applications}
	
	\author{Alessandro Cacco\\
		mat. 203345\\
		{\tt\small alessandro.cacco@studenti.unitn.it}
		\and
		Andrea Ferigo\\
		mat. 207486\\
		{\tt\small andrea.ferigo@studenti.unitn.it}
		\and
		Enrico Zardini\\
		mat. 207465\\
		{\tt\small enrico.zardini@studenti.unitn.it}
	}
	\date{}
	\maketitle
	
	\section{Introduction}
	\label{sec:intro}
	This work aims at illustrating an implementation of \textit{Chord}, a scalable distributed lookup protocol described in \cite{chord}. Basically, \textit{Chord} provides a primitive, i.e. \textit{lookup}, that allows to determine the responsible of a key in an efficient way. Hence, it represents a great solution to the data location problem: each data item needs just to be associated to a key and stored in the node the key maps to. \newline 
	Moreover, \textit{Chord} exploits consistent hashing to assign keys to nodes - in order to keep the load balanced - and requires that each node maintains information about only a few other nodes. Therefore, it scales well to large numbers of nodes  without affecting performance. Actually, \textit{Chord} adapts effectively also in dynamic environments with frequent joins and leaves thanks to a simple stabilization algorithm. \newline
	Starting from this, \cref{sec:implementation} will describe in detail the implementation, \cref{sec:simulator} will present the graphical simulator that has been developed to show the protocol's functioning and \cref{sec:analyses} will describe the simulations that have been performed and the results obtained. 
	
	\section{Implementation}
	\label{sec:implementation}
	
	\section{Simulator}
	\label{sec:simulator}
	
	\section{Analyses and Results}
	\label{sec:analyses}
	
	\printbibliography
\end{document}















































